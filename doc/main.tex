\documentclass[a4paper,10pt]{article}
\usepackage[utf8]{inputenc}
\usepackage{textcomp}
\usepackage{graphicx}
\usepackage{caption}
\usepackage{subcaption}
%\usepackage{subfigure}
\usepackage{parskip}
\usepackage{hyperref}
\usepackage{color}
\usepackage{pdfpages}
\usepackage{pgf}
\usepackage{tikz}
\usepackage{amsfonts}
\usepackage{float}
\usepackage{cleveref}
\usepackage{multirow}
\usepackage{pdflscape}
\usepackage{rotating}
\usepackage{amsmath}
\usepackage{tablefootnote}
\restylefloat{table}
\usetikzlibrary{arrows,automata}

\begin{document}
\begin{titlepage}
			\begin{center}
			\textsc{\LARGE {IN4391 - Distributed Computing Systems}}\\[1cm]
			\rule{\linewidth}{0.5mm} \\[0.4cm]

			{\Huge \bfseries Large Lab Exercise B}\\[0.15cm]

			\rule{\linewidth}{0.5mm} \\
			\textsc{\large{The Dragons Arena System}}
			
				\rule{\linewidth}{0.5mm}
			
			\vskip 0.5 cm
			
			\begin{minipage}{0.4\textwidth}
				\begin{flushleft} \large
					\emph{Authors:}\\
					\begin{tabular}{l}
						Q. Stokkink (xxxxxxx) \\
						J.E.T. Tan (4032918)\\
					\end{tabular}
				\end{flushleft}
			\end{minipage}
			\hspace{1cm}
			\begin{minipage}{0.4\textwidth}
				\begin{flushright} \large
					\emph{Support Cast:} \\
					Dr.Ir. A. Iousup \\
					Ir. Y. Guo
				\end{flushright}
			\end{minipage}
			
			\vskip 0.5 cm
			
			\rule{\linewidth}{0.5mm}
			\end{center}
\end{titlepage}

\section{Abstract}
\label{sec:abstract}

Here comes the abstract...

\section{Introduction}
\label{sec:intro}

Massive Multiplayer Online Games (MMOG) are games that require player-to-player or player-to-environment interactions with players all over the world.
Each player also have their own platform (Desktop, laptop, tablet, smartphone or game console) with different hardware and different performance capacities.
Furthermore, the amount of players online is not known beforehand.
Because of this, game developers cannot rely on a traditional central server anymore.
Distributed computing solutions are commonly used nowadays in Massive Multiplayer Online Games (MMOG) because of their reliability,
scalability and predictable high performance compared to having a central server.

In this report, we describe the Dragons Arena System's (DAS) game engine,
a distributed game engine written in Java that allows us to scale the system according to the amount of players and synchronizes the action of each player with multiple servers.
The remainder of this report is divided as follows: in section 3, we provide background information about DAS.
In section 4, we present our system design of the distributed game engine behind it.
We then present our experimental results in section 5 and discuss them in section 6.
Finally, we conclude the report in section 7.


\section{Background on Application}
\label{sec:background}

In this section, we will shortly describe background information about the DAS application and the requirements stated for the game engine behind it.

DAS is an online warfare in which the world is a grid battlefield of 25x25 squares.
In this battlefield, dragons and knights (players) battle each other until the strongest race survives.

To implement a distributed game engine of DAS, the following requirements have been proposed:

\begin{itemize}
	\item Since it is difficult to test the system with real players, it has to be emulated (i.e. players take a simple strategy where they heal a nearby player with low health or attack a dragon otherwise)
	\item Clients and server nodes could crash at any time and restart. To handle this, all events must be logged in the order in which they occur on at least 2 server nodes
	\item The DAS application must be tested with any number of players up to 100 and up to 20 dragons with 5 server nodes
\end{itemize}


\section{System Design}
\label{sec:design}
Design...

\section{Experimental Results}
\label{sec:results}
Here we present our experimental results.

\section{Discussion}
\label{sec:discussion}
Discuss main findings!

\section{Conclusion}
\label{sec:conclusion}
In this project...

\newpage
\section*{Appendix}

\subsection*{A - Time Sheet}
\label{sec:appendix}



\end{document}
